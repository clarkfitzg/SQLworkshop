% From https://www.sharelatex.com/learn/Typing_exams_in_LaTeX
% 
% To produce pdf run:
%   $ pdflatex paper.tex 
%

\documentclass[10pt]{exam}

\usepackage{listings}
\lstset{basicstyle=\footnotesize\ttfamily,breaklines=true}
 
% These do what you would expect:
\printanswers
%\noprintanswers

\begin{document}
 
\title{SQL Workshop Week 2}
\author{UC Davis DSI}
\date{May 11, 2018}

\maketitle

\section{Combining Data}

From Wikipedia:

    \begin{quote}
        The Standard Industrial Classification (SIC) is a system for
        classifying industries by a four-digit code. Established in the
        United States in 1937, it is used by government agencies to
        classify industry areas.
    \end{quote}

\begin{questions}

\question How many different SIC codes appear in the \texttt{company\_info}
table?

\begin{solution}
\begin{lstlisting}
SELECT COUNT(DISTINCT(sic_code))
FROM company_info
;
\end{lstlisting}
\end{solution}

    \question Which SIC codes appear in \texttt{company\_name} but have no
    description in the \texttt{sic} table?

\begin{solution}
\begin{lstlisting}
SELECT DISTINCT(sic_code)
FROM company_info
EXCEPT
SELECT SIC
FROM sic
;
\end{lstlisting}
\end{solution}


\question Select the ticker, company name, and SIC code with corresponding description for
    companies that have SIC codes.

\begin{solution}
\begin{lstlisting}
SELECT ticker, company_name, sic_code, Description
FROM company_info
INNER JOIN sic
ON company_info.sic_code = sic.SIC
;
\end{lstlisting}
\end{solution}

\question Select the ticker, company name, and SIC code with corresponding
    description (possibly NULL) for
    all companies in the \texttt{company\_info} table. 

\begin{solution}
\begin{lstlisting}
SELECT ticker, company_name, sic_code, Description
FROM company_info
LEFT JOIN sic
ON company_info.sic_code = sic.SIC
;
\end{lstlisting}
\end{solution}

\question Which 4 digit SIC codes have the most companies?

\begin{solution}
\begin{lstlisting}
SELECT sic_code, Description, count(*) as cnt
FROM company_info
LEFT JOIN sic
ON company_info . sic_code = sic . SIC
GROUP BY sic_code
ORDER BY cnt DESC
;
\end{lstlisting}
\end{solution}


\question Which states have the most companies in the
    \texttt{company\_info} table?

\begin{solution}
\begin{lstlisting}
SELECT state, COUNT(*) as cnt
FROM company_locations
GROUP BY state
ORDER BY cnt DESC
;
\end{lstlisting}
\end{solution}


\question Write a query that produces a table with columns for state, SIC
code, SIC description, and count of companies located in that state
with that SIC code.

\begin{solution}
\begin{lstlisting}
SELECT l.state,  i.sic_code, s.Description, count(*) as cnt
FROM company_info as i
LEFT JOIN sic as s
ON i.sic_code = s.SIC
INNER JOIN company_locations as l
ON l.ticker = i.ticker
GROUP BY i.sic_code
ORDER BY l.state ASC, cnt DESC
;
\end{lstlisting}
\end{solution}


\question Modify the query above to only return rows with a count
of three or more companies.

\begin{solution}
\begin{lstlisting}
SELECT l.state,  i.sic_code, s.Description, count(*) as cnt
FROM company_info as i
LEFT JOIN sic as s
ON i.sic_code = s.SIC
INNER JOIN company_locations as l
ON l.ticker = i.ticker
GROUP BY i.sic_code
HAVING cnt >= 3
ORDER BY l.state ASC, cnt DESC
;
\end{lstlisting}
\end{solution}



%    \question Which ticker symbols appear in both
%    \texttt{company\_locations} and \texttt{daily\_share\_prices}?
%
%\begin{solution}
%\begin{lstlisting}
%SELECT ticker
%FROM company_locations
%INTERSECT
%SELECT ticker
%FROM daily_share_prices
%;
%\end{lstlisting}
%\end{solution}



\end{questions}
\end{document}
