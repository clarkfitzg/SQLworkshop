% From https://www.sharelatex.com/learn/Typing_exams_in_LaTeX
% 
% To produce pdf run:
%   $ pdflatex paper.tex 
%

\documentclass[10pt]{exam}

\usepackage{listings}
\lstset{basicstyle=\footnotesize\ttfamily,breaklines=true}
 
% These do what you would expect:
\printanswers
%\noprintanswers

\begin{document}
 
\title{SQL Workshop Week 2}
\author{UC Davis DSI}
\date{2018}

\maketitle

\section{Review}

\begin{questions}

\question Which tickers have a beta of 3 or more?

\begin{solution}
\begin{lstlisting}
SELECT *
FROM financial_ratios
WHERE beta >= 3
;
\end{lstlisting}
\end{solution}


\question How many days in 2014-2015 did Amazon (AMZN) shares close above 350?

\begin{solution}
\begin{lstlisting}
SELECT COUNT(*)
FROM daily_share_prices
WHERE
	ticker = 'AMZN'
	AND date BETWEEN '2014-01-01' AND '2015-01-01'
	AND close > 350
;
\end{lstlisting}
\end{solution}


\question Are any values missing from the state populations table?

\begin{solution}
\begin{lstlisting}
SELECT *
FROM state_populations
WHERE
	popnEstimate2017 IS NULL
	OR popn2010 IS NULL
;
\end{lstlisting}
\end{solution}


\end{questions}


%%%%%%%%%%%%%%%%%%%%%%%%%%%%%%%%%%%%%%%%%%%%%%%%%%%%%%%%%%%%%%%%%%%%%%%%%%%%%%%
\clearpage

\section{Aggregation}

\begin{questions}

%%%%% Aggregate Functions %%%%%
\subsection*{Aggregation Functions}

\question What's the total assets of all companies in the company info table?

\begin{solution}
\begin{lstlisting}
SELECT SUM(asset)
FROM company_info
;
\end{lstlisting}
\end{solution}


\question How many companies are in the "Energy" sector?

\begin{solution}
\begin{lstlisting}
SELECT COUNT(*)
FROM company_info
WHERE sector = 'Energy'
;
\end{lstlisting}
\end{solution}


\question Which ticker has the most assets? Which has the least? Answer this
question without using \lstinline{ORDER BY}.

\begin{solution}
\begin{lstlisting}
-- Min assets
SELECT ticker, min(asset)
FROM company_info
;

-- Max assets
SELECT ticker, max(asset)
FROM company_info
;
\end{lstlisting}
\end{solution}


\question What happens if you aggregate two different columns in the same
query? Give an example and explain the result. What happens if you also include
columns that aren't aggregated?

\begin{solution}
Consider the query
\begin{lstlisting}
SELECT ticker, min(asset), max(asset)
FROM company_info
;
\end{lstlisting}
The result of this query is
\begin{lstlisting}
ticker  min(asset)  max(asset)
FNMA    74100.0     3287968000000.0
\end{lstlisting}
The company FNMA has the most assets and FTI has the least. The query returns
the correct min and max values for the asset column, but only the ticker FNMA.

By trying several different queries, it looks like rows for columns that aren't
aggregated are selected based on the last aggregation in the query. In the
example query, that's \lstinline{max(asset)}, so FNMA is selected rather than
FTI.
\end{solution}



%%%%% GROUP BY %%%%%
\subsection*{Grouping}

\question How many Fortune 500 companies are in each state?

\begin{solution}
\begin{lstlisting}
SELECT state, COUNT(*) AS num_companies
FROM company_locations
GROUP BY state
ORDER BY num_companies DESC
;
\end{lstlisting}
\end{solution}


\question According to the company locations table, how many companies went
down in Fortune 500 rankings from 2014 to 2015? How many went up? Try to get
all of this information in one query.

\begin{solution}
\begin{lstlisting}
SELECT trend, COUNT(*)
FROM company_locations
GROUP BY trend
;
\end{lstlisting}
\end{solution}


\question What is the average opening price for each stock in the daily share
prices table? Sort the result from lowest to highest average price.

\begin{solution}
\begin{lstlisting}
SELECT ticker, AVG(open) AS avg_price
FROM daily_share_prices
GROUP BY ticker
ORDER BY avg_price
;
\end{lstlisting}
\end{solution}


\question Which companies in the daily share prices table had the 3 highest
average closing prices for 2014? Hint: use \lstinline{STRFTIME('%Y', date)} to
extract the year.

\begin{solution}
\begin{lstlisting}
SELECT ticker, AVG(close) AS avg_close
FROM daily_share_prices
WHERE STRFTIME('%Y', date) = '2014'
GROUP BY ticker
ORDER BY avg_close DESC
LIMIT 3
;
\end{lstlisting}
\end{solution}



%%%%% HAVING %%%%%
\subsection*{Filter Rows After Aggregation}

\question Which states have between 5 and 10 companies?

\begin{solution}
\begin{lstlisting}
SELECT state, COUNT(*) AS freq
FROM company_locations
GROUP BY state
HAVING freq BETWEEN 5 AND 10
ORDER BY freq
;
\end{lstlisting}
\end{solution}


\question Which stocks have negative average daily increase ($\textrm{close}
- \textrm{open})$) over the 5-year period?

\begin{solution}
\begin{lstlisting}
SELECT ticker, AVG(close - open) AS avg_increase
FROM daily_share_prices
GROUP BY ticker
HAVING avg_increase < 0
;
\end{lstlisting}
\end{solution}


\question Which year and month combinations (e.g., April 2017, June 2016) have
less than 8000 share prices? Hint: use \lstinline{STRFTIME('%Y-%m', date)} to
extract the year and month.

\begin{solution}
\begin{lstlisting}
SELECT
	STRFTIME('%Y-%m', date) AS year_month,
	COUNT(*) AS freq
FROM daily_share_prices
GROUP BY year_month
HAVING freq < 8000
;
\end{lstlisting}
\end{solution}


\question Which industries have a negative average earning per share?

\begin{solution}
\begin{lstlisting}
SELECT sector, AVG(earning_per_share) AS avg_earning
FROM company_info
GROUP BY industry
HAVING avg_earning <= 0
ORDER BY avg_earning
;
\end{lstlisting}
\end{solution}



\end{questions}

%%%%%%%%%%%%%%%%%%%%%%%%%%%%%%%%%%%%%%%%%%%%%%%%%%%%%%%%%%%%%%%%%%%%%%%%%%%%%%%
\clearpage

\section{Combining Data}

From Wikipedia:

    \begin{quote}
        The Standard Industrial Classification (SIC) is a system for
        classifying industries by a four-digit code. Established in the
        United States in 1937, it is used by government agencies to
        classify industry areas.
    \end{quote}

\begin{questions}

\question How many different SIC codes appear in the \texttt{company\_info}
table?

\begin{solution}
\begin{lstlisting}
SELECT COUNT(DISTINCT(sic_code))
FROM company_info
;
\end{lstlisting}
\end{solution}

    \question Which SIC codes appear in \texttt{company\_name} but have no
    description in the \texttt{sic} table?

\begin{solution}
\begin{lstlisting}
SELECT DISTINCT(sic_code)
FROM company_info
EXCEPT
SELECT SIC
FROM sic
;
\end{lstlisting}

We can also use a subquery
\begin{lstlisting}
SELECT DISTINCT(sic_code)
FROM company_info
WHERE NOT sic_code IN (SELECT SIC FROM sic);
\end{lstlisting}
\end{solution}



\subsection*{Single Joins}

\question Select the ticker, company name, and SIC code with corresponding description for
    companies that have SIC codes.

\begin{solution}
\begin{lstlisting}
SELECT ticker, company_name, sic_code, Description
FROM company_info
INNER JOIN sic
ON company_info.sic_code = sic.SIC
;
\end{lstlisting}
\end{solution}


\question Select the ticker, company name, and SIC code with corresponding
    description (possibly NULL) for
    all companies in the \texttt{company\_info} table. 

\begin{solution}
\begin{lstlisting}
SELECT ticker, company_name, sic_code, Description
FROM company_info
LEFT JOIN sic
ON company_info.sic_code = sic.SIC
;
\end{lstlisting}
\end{solution}

\question Which 4 digit SIC codes have the most companies?

\begin{solution}
\begin{lstlisting}
SELECT sic_code, Description, count(*) as cnt
FROM company_info
LEFT JOIN sic
ON company_info . sic_code = sic . SIC
GROUP BY sic_code
ORDER BY cnt DESC
;
\end{lstlisting}
\end{solution}


\subsection*{Sub-Queries}

\question
Focusing only on 2014 and the \texttt{daily\_share\_prices} table,
find the names of the companies which had any closing price that exceed the average closing price of AAPL in 2014.


\begin{solution}
This is a two-step operation.
We need to first compute the average closing price for Apple stock
for 2014.
We do this code we have already practiced:
\begin{lstlisting}
SELECT AVG(close) FROM daily_share_prices
WHERE ticker = 'AAPL' AND date BETWEEN '2014-01-01' AND '2014-12-31';
\end{lstlisting}

We could read the average price we get and use it to
filter rows in a new query by typing it directly.
But that is not ideal. Instead, we'll compute this
result and use it directly in our larger query.

\begin{lstlisting}
SELECT ticker FROM daily_share_prices
WHERE
   date BETWEEN '2014-01-01' AND '2014-12-31'
 AND
   close > (SELECT AVG(close) FROM daily_share_prices
             WHERE ticker = 'AAPL' AND date BETWEEN '2014-01-01' AND '2014-12-31');
\end{lstlisting}
This gives back many rows, the first of which are AAPL.
What we have is a row for each day that a close price exceeded the average.
We want just the names of the companies, so not the same name repeated
as many times as that ticker exceeded the AAPL average.
We do this by adding DISTINCT to our query to make the rows unique:
\begin{lstlisting}
SELECT DISTINCT ticker FROM daily_share_prices
WHERE
   date BETWEEN '2014-01-01' AND '2014-12-31'
 AND
   close > (SELECT AVG(close) FROM daily_share_prices
             WHERE ticker = 'AAPL' AND date BETWEEN '2014-01-01' AND '2014-12-31');
\end{lstlisting}
\end{solution}

Bonus Question  Count the number of days these stocks had a close price above this average.

\begin{solution}
For this, we use a GROUP BY and a call to COUNT() to get the result we want,
discarding the DISTINCT we added above for a different purpose:
\begin{lstlisting}  
SELECT ticker, COUNT(ticker) FROM daily_share_prices
WHERE
   date BETWEEN '2014-01-01' AND '2014-12-31'
 AND
   close > (SELECT AVG(close) FROM daily_share_prices
             WHERE ticker = 'AAPL' AND date BETWEEN '2014-01-01' AND '2014-12-31')
GROUP BY ticker;
\end{lstlisting}
\end{solution}


\subsection*{Multiple Joins}

\question Write a query that produces a table with columns for state, SIC
code, SIC description, and count of companies located in that state
with that SIC code.

% 
\begin{solution}
\begin{lstlisting}
SELECT l.state,  i.sic_code, s.Description, count(*) as cnt
FROM company_info as i
LEFT JOIN sic as s
ON i.sic_code = s.SIC
INNER JOIN company_locations as l
ON l.ticker = i.ticker
GROUP BY i.sic_code, l.state
ORDER BY l.state ASC, cnt DESC
;
\end{lstlisting}


We'll break this into 2 steps.
We want the state and ticker and SIC ode all in one table
for each company. If a compay doesn't have a SIC code, we want to drop.
So this an INNER JOIN.
\begin{lstlisting}
SELECT l.ticker, l.state, r.sic_code
FROM company_locations AS l
INNER JOIN company_info AS r
ON l.ticker = r.ticker;
\end{lstlisting}

Having done this, we can do the next step.
This merges the description of the SIC code 

Now with this, we want to count by


\begin{lstlisting}
SELECT state, sic_code, COUNT(ticker) AS cnt, B.description
FROM 
   (SELECT l.ticker, l.state, r.sic_code
        FROM company_locations AS l
        INNER JOIN company_info AS r
        ON l.ticker = r.ticker )
      AS A
LEFT JOIN sic as B
ON A.sic_code = B.SIC
GROUP BY state, sic_code
ORDER BY state ASC, cnt DESC;
\end{lstlisting}



\end{solution}


\question Modify the query above to only return rows with a count
of three or more companies.

\begin{solution}
\begin{lstlisting}
SELECT l.state,  i.sic_code, s.Description, count(*) as cnt
FROM company_info as i
LEFT JOIN sic as s
ON i.sic_code = s.SIC
INNER JOIN company_locations as l
ON l.ticker = i.ticker
GROUP BY i.sic_code, l.state
HAVING cnt >= 3
ORDER BY l.state ASC, cnt DESC
;
\end{lstlisting}
\end{solution}



    \question Which ticker symbols appear in both
    \texttt{company\_locations} and \texttt{daily\_share\_prices}?

\begin{solution}
 We can use a subquery again.
\begin{lstlisting}
SELECT ticker
FROM company_locations
WHERE ticker IN (SELECT ticker FROM daily_share_prices);
\end{lstlisting}

Alternatively, like EXCEPT, we can use INTERSECT
\begin{lstlisting}
SELECT ticker
FROM company_locations
INTERSECT
SELECT ticker
FROM daily_share_prices
;
\end{lstlisting}
\end{solution}


\end{questions}
\end{document}
