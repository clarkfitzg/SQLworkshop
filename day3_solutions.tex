% From https://www.sharelatex.com/learn/Typing_exams_in_LaTeX
% 
% To produce pdf run:
%   $ pdflatex paper.tex 
%

\documentclass[10pt]{exam}

\usepackage{listings}
\lstset{basicstyle=\footnotesize\ttfamily,breaklines=true}
 
% These do what you would expect:
\printanswers
%\noprintanswers

\begin{document}
 
\title{SQL Workshop Week 3}
\author{UC Davis DSI}
\date{May 18, 2018}

\maketitle

\section{Review}

\begin{questions}

\question What are the most valuable sectors?

\begin{solution}
\begin{lstlisting}
SELECT
\end{lstlisting}
\end{solution}

\end{questions}
\section{Joins}

From Wikipedia:

    \begin{quote}
        The Standard Industrial Classification (SIC) is a system for
        classifying industries by a four-digit code. Established in the
        United States in 1937, it is used by government agencies to
        classify industry areas.
    \end{quote}

\begin{questions}

\subsection*{Single Joins}

\question Select the ticker, company name, and SIC code with corresponding description for
    companies that have SIC codes.

\begin{solution}
\begin{lstlisting}
SELECT ticker, company_name, sic_code, Description
FROM company_info
INNER JOIN sic
    ON company_info.sic_code = sic.SIC
;
\end{lstlisting}
\end{solution}


\question Select the ticker, company name, and SIC code with corresponding
    description (possibly NULL) for
    all companies in the \texttt{company\_info} table. 

\begin{solution}
\begin{lstlisting}
SELECT ticker, company_name, sic_code, Description
FROM company_info
LEFT JOIN sic
    ON company_info.sic_code = sic.SIC
;
\end{lstlisting}
\end{solution}

\question Which 4 digit SIC codes have the most companies?

\begin{solution}
\begin{lstlisting}
SELECT sic_code, Description, count(*) as cnt
FROM company_info
LEFT JOIN sic
    ON company_info . sic_code = sic . SIC
GROUP BY sic_code
ORDER BY cnt DESC
;
\end{lstlisting}
\end{solution}

\question Modify the query above to only return SIC codes with at least
5 companies.

\begin{solution}
\begin{lstlisting}
SELECT sic_code, Description, count(*) as cnt
FROM company_info
LEFT JOIN sic
    ON company_info . sic_code = sic . SIC
GROUP BY sic_code
HAVING cnt >= 5
ORDER BY cnt DESC
;
\end{lstlisting}
\end{solution}



\subsection*{Multiple Joins}

\question Use inner joins to produce a table with columns for company name,
state, SIC code, and SIC description.

\begin{solution}
\begin{lstlisting}
SELECT n.sandp_company_name, l.state,  i.sic_code, s.Description
FROM company_info as i
INNER JOIN company_name as n
    ON n.ticker = i.ticker
INNER JOIN sic as s
    ON i.sic_code = s.SIC
INNER JOIN company_locations as l
    ON l.ticker = i.ticker
;
\end{lstlisting}

Or you may prefer this syntax:

\begin{lstlisting}
SELECT n.sandp_company_name, l.state,  i.sic_code, s.Description
FROM company_info as i
    , company_name as n
    , sic as s 
    , company_locations as l 
WHERE n.ticker = i.ticker
AND i.sic_code = s.SIC
AND l.ticker = i.ticker
;
\end{lstlisting}
\end{solution}


\question Modify the above query to 
produce a table with columns for state, SIC
code, SIC description, and count of companies located in that state
with that SIC code.

\begin{solution}

    We no longer need company name, so we can remove the join to that
    table. Then we add a \texttt{GROUP BY} clause.

\begin{lstlisting}
SELECT l.state,  i.sic_code, s.Description, count(*) as cnt
FROM company_info as i
INNER JOIN sic as s
    ON i.sic_code = s.SIC
INNER JOIN company_locations as l
    ON l.ticker = i.ticker
GROUP BY i.sic_code, l.state
ORDER BY l.state ASC, cnt DESC
;
\end{lstlisting}
\end{solution}


\question Modify the query above to only return rows with a count
of at least 2 companies.

\begin{solution}
\begin{lstlisting}
SELECT l.state,  i.sic_code, s.Description, count(*) as cnt
FROM company_info as i
INNER JOIN sic as s
    ON i.sic_code = s.SIC
INNER JOIN company_locations as l
    ON l.ticker = i.ticker
GROUP BY i.sic_code, l.state
HAVING cnt >= 2
ORDER BY l.state ASC, cnt DESC
;
\end{lstlisting}
\end{solution}


\end{questions}
\end{document}
