% From https://www.sharelatex.com/learn/Typing_exams_in_LaTeX
% 
% To produce pdf run:
%   $ pdflatex paper.tex 
%

\documentclass[10pt]{exam}

\usepackage{listings}
\lstset{basicstyle=\footnotesize\ttfamily,breaklines=true}
 
% These do what you would expect:
%\printanswers
\noprintanswers

\begin{document}
 
\title{SQL Workshop Week 3}
\author{UC Davis DSI}
\date{May 18, 2018}

\maketitle

\section{Review}

    These questions are meant to be more open ended. Challenge yourself by
    answering each one in two different ways. This requires three types of
    knowledge: knowledge of the data / domain (finance), knowledge of the
    tool (SQL), and critical thinking. This is the next step beyond following
    instructions.

\begin{questions}

\question What are the most profitable companies?

\begin{solution}

1. We can compute standard metrics that don't exist in the database such as
return on assets (ROA).
\texttt{DISTINCT} addresses a data quality issue by removing the duplicated
    rows.

\begin{lstlisting}
SELECT DISTINCT company_name, net_income / asset as return_on_assets
FROM company_info
ORDER BY  return_on_assets DESC
;
\end{lstlisting}

2.
Return on equity directly measures profitability for investors.

\begin{lstlisting}
SELECT cn.sandp_company_name, fr.return_on_equity
FROM company_name as cn
    INNER JOIN financial_ratios as fr
ON cn.ticker = fr.ticker
ORDER BY fr.return_on_equity DESC
;
\end{lstlisting}

\end{solution}


\question What are the most valuable sectors?

\begin{solution}
    This is a question about sectors, so you should be thinking about
    aggregating figures up to the sector level, which means using \texttt{GROUP
    BY sector}. Use \texttt{ORDER BY} so that the most valuable show up on
    top.

1.    Showing the assets in each sector in trillions of dollars rather than single units lets us
    readily read and compare them.

\begin{lstlisting}
SELECT sector, SUM(asset) / 1e12 as trillions
FROM company_info
GROUP BY sector
ORDER BY trillions DESC
;
\end{lstlisting}

2. This computes the average earnings per share in each industry

\begin{lstlisting}
SELECT sector, AVG(earning_per_share) as avg_earning_per_share
FROM company_info
GROUP BY sector
ORDER BY  avg_earning_per_share DESC
;
\end{lstlisting}
\end{solution}


\end{questions}


%%%%%%%%%%%%%%%%%%%%%%%%%%%%%%%%%%%%%%%%%%%%%%%%%%%%%%%%%%%%
\section{Joins}

From Wikipedia:

    \begin{quote}
        The Standard Industrial Classification (SIC) is a system for
        classifying industries by a four-digit code. Established in the
        United States in 1937, it is used by government agencies to
        classify industry areas.
    \end{quote}

\begin{questions}

\subsection*{Single Joins}

\question Select the ticker, company name, and SIC code with corresponding description for
    companies that have SIC codes.

\begin{solution}
\begin{lstlisting}
SELECT ticker, company_name, sic_code, Description
FROM company_info
INNER JOIN sic
    ON company_info.sic_code = sic.SIC
;
\end{lstlisting}
\end{solution}


\question Select the ticker, company name, and SIC code with corresponding
    description (possibly NULL) for
    all companies in the \texttt{company\_info} table. 

\begin{solution}
\begin{lstlisting}
SELECT ticker, company_name, sic_code, Description
FROM company_info
LEFT JOIN sic
    ON company_info.sic_code = sic.SIC
;
\end{lstlisting}
\end{solution}

\question Which 4 digit SIC codes have the most companies?

\begin{solution}
\begin{lstlisting}
SELECT sic_code, Description, count(*) as cnt
FROM company_info
LEFT JOIN sic
    ON company_info.sic_code = sic.SIC
GROUP BY sic_code
ORDER BY cnt DESC
;
\end{lstlisting}
\end{solution}

\question Modify the query above to only return SIC codes with at least
5 companies.

\begin{solution}
\begin{lstlisting}
SELECT sic_code, Description, count(*) as cnt
FROM company_info
LEFT JOIN sic
    ON company_info.sic_code = sic.SIC
GROUP BY sic_code
HAVING cnt >= 5
ORDER BY cnt DESC
;
\end{lstlisting}
\end{solution}



\subsection*{Multiple Joins}

\question Use inner joins to produce a table with columns for company name,
state, SIC code, and SIC description.

\begin{solution}
\begin{lstlisting}
SELECT n.sandp_company_name, l.state, i.sic_code, s.Description
FROM company_info as i
INNER JOIN company_name as n
    ON n.ticker = i.ticker
INNER JOIN sic as s
    ON i.sic_code = s.SIC
INNER JOIN company_locations as l
    ON l.ticker = i.ticker
;
\end{lstlisting}

Or you may prefer this syntax:

\begin{lstlisting}
SELECT n.sandp_company_name, l.state, i.sic_code, s.Description
FROM company_info as i
    , company_name as n
    , sic as s 
    , company_locations as l 
WHERE n.ticker = i.ticker
AND i.sic_code = s.SIC
AND l.ticker = i.ticker
;
\end{lstlisting}
\end{solution}


\question Modify the above query to 
produce a table with columns for state, SIC
code, SIC description, and count of companies located in that state
with that SIC code.

\begin{solution}

    We no longer need company name, so we can remove the join to that
    table. Then we add a \texttt{GROUP BY} clause.

\begin{lstlisting}
SELECT l.state,  i.sic_code, s.Description, count(*) as cnt
FROM company_info as i
INNER JOIN sic as s
    ON i.sic_code = s.SIC
INNER JOIN company_locations as l
    ON l.ticker = i.ticker
GROUP BY i.sic_code, l.state
ORDER BY l.state ASC, cnt DESC
;
\end{lstlisting}
\end{solution}


\question Modify the query above to only return rows with a count
of at least 2 companies.

\begin{solution}
\begin{lstlisting}
SELECT l.state,  i.sic_code, s.Description, count(*) as cnt
FROM company_info as i
INNER JOIN sic as s
    ON i.sic_code = s.SIC
INNER JOIN company_locations as l
    ON l.ticker = i.ticker
GROUP BY i.sic_code, l.state
HAVING cnt >= 2
ORDER BY l.state ASC, cnt DESC
;
\end{lstlisting}
\end{solution}

\end{questions}


%%%%%%%%%%%%%%%%%%%%%%%%%%%%%%%%%%%%%%%%%%%%%%%%%%%%%%%%%%%%
\section{Putting It All Together}

\begin{questions}

\question Did the Brexit vote affect share prices for commercial banks in the
United States? How?

Use Google Sheets to show this with a plot.

\begin{solution}
First we need to find out when the Brexit vote took place. We can look online
to find June 23, 2016. So we need to get share prices around this time.

We also need to identify the commercial banks. There are several ways to do
this. We can make a list of banks, or use SIC codes (since the codes identify
each company's industry). To find out the right SIC codes, we can look online
or we examine a well-known bank in the database.

For example, the ticker for Citigroup is ``C'', so we can run the query
\begin{lstlisting}
SELECT *
FROM company_info
WHERE ticker = 'C'
;
\end{lstlisting}
This gives us SIC code 6021. We can check all the companies with that SIC code
with the query
\begin{lstlisting}
SELECT *
FROM company_info
WHERE sic_code = 6021
;
\end{lstlisting}
The results appear to all be banks, so we can use 6021, but what if some banks
use a different SIC code? To check this, we can look for companies in the
``Banks'' industry that have a different SIC code. A query to do this is
\begin{lstlisting}
SELECT *
FROM company_info
WHERE industry = 'Banks'
  AND sic_code != 6021
;
\end{lstlisting}
This leads to three banks: Fannie Mae, Fifth Third Bancorp, and Citizens
Financial. Fannie Mae is not a commercial bank (we can find this information
online, for example on Wikipedia). Both Fifth Third Bancorp and Citizens
Financial are commercial banks, and both have SIC code 6022. So we should also
check 6022.
\begin{lstlisting}
SELECT *
FROM company_info
WHERE sic_code = 6022
;
\end{lstlisting}
This query finds two more banks. So we can use all the companies listed under
SIC code 6021 and 6022 as banks.

An even more careful strategy is to search for SIC codes that mention banks in
the sic table.

Now we can get relevant share prices. First we select the tickers for banks:
\begin{lstlisting}
SELECT ticker
FROM company_info
WHERE sic_code IN (6021, 6022)
;
\end{lstlisting}
Now we use this query as a subquery to get all the share prices around June 23,
2016:
\begin{lstlisting}
SELECT date, ticker, open, close
FROM daily_share_prices
WHERE ticker IN (
  SELECT ticker
  FROM company_info
  WHERE sic_code IN (6021, 6022)
)
AND date BETWEEN '2016-06-16' AND '2016-07-23'
;
\end{lstlisting}
We could've used \texttt{JOIN} to find these instead of a subquery. This
has the advantage that we could also include
information that isn't in the daily share prices table (such as company names).
Here's what the join looks like:
\begin{lstlisting}
SELECT date, dsp.ticker, open, close
FROM company_info
JOIN daily_share_prices as dsp
    ON company_info.ticker = dsp.ticker
WHERE sic_code IN (6021, 6022)
AND date BETWEEN '2016-06-16' AND '2016-07-23'
;
\end{lstlisting}

Now we can save the share prices from our query to a comma-seperated values
(CSV) file using the menus in the database browser. Google Sheets and Microsoft
Excel can read CSV files, so we can use either to create plots of the data.
\end{solution}


\question Based on the information in the database, what are the dominant
industries in New York? How does New York compare to other states?

\begin{solution}
First we need to find where each company is located, so we can look at the
company locations table. Since that table doesn't have the industry for each
company, we need to join its information to the company info table. So we can
write the first draft of our query as
\begin{lstlisting}
SELECT L.ticker, L.company, L.state, R.industry
FROM
  company_location AS L
LEFT JOIN
  company_info as R
ON L.ticker = R.ticker
\end{lstlisting}

Since the goal is to compare industries in different states, we need
information per state and industry, rather than per company. So we can change
the query to group by state and industry to count the number of companies.
Since we are aggregating, it no longer makes sense to keep the ticker and
company name. The second draft of our query is
\begin{lstlisting}
SELECT L.state, R.industry, COUNT(*) as freq
FROM
  company_location AS L
LEFT JOIN
  company_info as R
ON L.ticker = R.ticker
GROUP BY state, industry
\end{lstlisting}

Finally, the result is a lot easier to read if we sort by state and by the
number of companies within each state. The third draft of our query is
\begin{lstlisting}
SELECT L.state, R.industry, COUNT(*) as freq
FROM
  company_locations AS L
LEFT JOIN
  company_info as R
ON L.ticker = R.ticker
GROUP BY state, industry
ORDER BY state, freq DESC
\end{lstlisting}

This shows that New York state doesn't really seem to have a dominant industry,
but there are 2 companies each for apparel, investment banking, insurance,
and pharmaceuticals. There are also 2 companies with no industry listed (NULL)
in the database. Of course, the database we're using only has a small amount of
data so it probably doesn't reflect the real situation in New York.

Most other states also have only 1 or 2 companies in each industry. We can sort
by number of companies (but not state) to see if any states do have a dominant
industry. The query to do this is
\begin{lstlisting}
SELECT L.state, R.industry, COUNT(*) as freq
FROM
  company_locations AS L
LEFT JOIN
  company_info as R
ON L.ticker = R.ticker
GROUP BY state, industry
ORDER BY freq DESC
\end{lstlisting}
The result shows that the oil industry is dominant in Texas (8 companies),
while internet services and computer hardware are dominant in California (7
companies). California also has 5 companies with no industry listed. No other
states show clearly dominant industries.
\end{solution}


\end{questions}
\end{document}
